\chapter{Conclusion}

In our research for this dissertation, we explored the initial steps needed to map from reactive programs to the dataflow execution model. 
These steps included creating our own experimental language FrDataFlow, writing a mapping algorithm from a reactive graph to a dataflow graph and executing the latter atop a data flow engine as described in \cite{saey_extensible_2017}. This chapter will revisit the original problems we set out to tackle, the solutions we proposed and conclude with some closing remarks. 

\section{Revisiting the problem statement}

Reactive programming is an event-first programming paradigm well suited for reactive systems (e.g. user interfaces, robotics, etc.) that provides the concept of signals which can be composed using declarative operators.
These signals are a common abstraction over I/0, events and asynchronous computations and provide a single interface to model these concepts.
Efficiency and timing is essential to create responsive and reliable reactive systems. However, the majority of reactive programming implementations are not designed to support parallelism, which is necessary to maximally utilise the processing power of modern processors. We propose to run reactive programs atop the dataflow execution model to make maximal use of the parallelization abilities provided by the underlying host. This execution model invokes instructions when the inputs are available and does not provide any form of global state, allowing instructions that do not share data dependencies to be invoked side by side. 

\section{Contributions}

We implemented an experimental Racket based language called FrDataFlow that adds reactive concepts to a subset of Racket constructs. This language was implemented using two different interpreters: one which directly implements the reactive mechanism in Racket and serves as a reactive runtime by manually keeping the graph of signals up to date using an update loop, as described in chapter 3 - Language.
The second implementation delegates the responsibility of keeping the reactive graph up to data to a dataflow engine implementation by mapping the signals to dataflow instructions in such a way that the signals behave the same as in a regular reactive system. 

More specifically, we provided the following contributions:

\begin{description}[style=nextline]
	\item[An interpreter for a reactive language] We designed and implemented an interpreter based on FrTime, but which chooses to provide explicit reactive operators rather than implicitly applying reactivity. This interpreter applies a traditional evaluation model to keep signals up to date, in the form of a signal update graph. 
	\item[A mapping algorithm from signals to dataflow instructions] We implemented an algorithm that maps the concepts of reactive programming to instructions in the dataflow execution model and solves a major mismatch between the two: the consumption of arguments in dataflow, as described in Chapter 4. Furthermore, we showed that reactive programs can be correctly translated to dataflow instructions so that they behave exactly the same. 
	\item[Comparison of a traditional reactive evaluation model and signals atop the dataflow execution model] We evaluated the performance of both approaches in terms of latency, throughput and scalability and showed that the platform overhead of parallelism, for now, does not outweigh the benefits of parallelism. We do note however that single threaded dataflow execution already brings with it benefits in terms of throughput, although at the cost of latency.
\end{description}

\newpage
\section{Final remarks}

The ultimate goal of this dissertation was to explore the possibilities of executing reactive programs atop a dataflow engine, in the hopes of achieving highly parallel execution of these programs. 

We notice that high level paradigms such as reactive programming are becoming increasingly popular, but they usually come with a performance cost due to these abstractions. In reaction to this, we tried to address these concerns by proving that the abstractions reactive programming makes can be correctly translated to a horizontally scalable execution model, namely a dataflow engine. In doing so, we touched on an unexplored subject in this space, making a meaningful contribution to both the world of reactive programming and the dataflow execution model. 

We look forward to seeing the further evolution of both in the future and the new insights they bring to the world of software development.  