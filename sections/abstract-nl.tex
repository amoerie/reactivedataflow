% !TeX spellcheck = nl_NL
\section*{Samenvatting}

Programma's in domeinen zoals robotica, het web en mobiele applicaties worden allemaal aangedreven door events. Dit inherent asynchrone model conflicteert met de synchrone aard van imperatief, sequentieel programmeren, en hierdoor zijn andere paradigma's verschenen om deze problemen beter aan te pakken. Een dergelijk paradigma dat speciaal ontworpen is voor \textit{event-driven} omgevingen is reactief programmeren, die events concretiseert als tijdvariabele waarden en ze samenstelbaar maakt aan de hand van declaratieve operatoren. Reactief programmeren probeert de uitdagingen van asynchroniciteit aan te pakken, zoals bijvoorbeeld de toepasselijk genaamde \textit{callback hell}, door een uniforme interface te voorzien voor verschillende asynchrone bronnen (e.g. I/O, events, ...) en op die manier te stroomlijnen hoe data doorgegeven wordt doorheen deze programma's.

Met dit hoge niveau van abstractie komt echter ook een bedenking met betrekking tot efficiëntie en timing, die van essentieel belang zijn voor deze systemen. Het merendeel van reactieve programma's tot nu toe zijn single threaded en benutten niet het volledige potentieel van hun onderliggende systemen. Onderzoek is reeds uitgevoerd om deze programma's te parallelliseren, maar wij stellen een ander alternatief voor: het mappen van reactieve programma's naar het dataflow executiemodel. Dit low level platform bepaalt dat primitieve instructies kunnen worden opgeroepen wanneer hun inputs aanwezig zijn, op voorwaarde dat ze geen gedeelde state uitlezen of muteren. Hierdoor kunnen niet-gerelateerde instructies (d.w.z. die geen data afhankelijkheden delen) parallel worden uitgevoerd.

In deze paper beschrijven we het proces van de vertaalslag van reactieve tijdsvariabele waarden naar instructies in het dataflow executiemodel. We presenteren twee interpreters voor een kleine experimentele taal speciaal ontworpen voor dit onderzoek: een eerste die reactiviteit in een eerder traditionele zin implementeert en een tweede die bovenop een dataflow engine draait. Daarnaast geven we een samenvatting van de mismatch tussen de twee modellen en bespreken we oplossingen hiervoor.

Tenslotte evalueren we de prestatie van beide benaderingen op vlak van efficiëntie en bespreken we de schaalbaarheid van reactieve programma's bovenop een dataflow engine. 